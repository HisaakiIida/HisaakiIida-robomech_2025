\documentclass{jarticle}
\usepackage{robomech}
\usepackage{graphicx}
\usepackage{amsmath}
\usepackage{bm}
\usepackage{here}
\usepackage{siunitx}

\begin{document}
\makeatletter
\title{飛行ロボットによるパーチングと物体把持を両立可能にするトライフォース・ハンドの設計と機体制御}
{Design and control of the tri-force hand for perching and object grasping with aerial robots}
% {―日本語副題:ゴシック体・12pt(欧文Arial・12pt)―}
{}
{}
% {-English Subtitle: Times New Roman, 10pt-}

\author{
\begin{tabular}{ll}
 \hspace{1zw}○ 学\hspace{1zw}飯田央明 (東京大)& 正\hspace{1zw} 趙漠居(東京大) \\ \hspace{1zw} 杉原淳一朗(東京大)& \hspace{1zw} 杉原和輝(東京大) \\ \hspace{1zw} 小塚陽希(東京大)& \hspace{1zw} 李謹傑(東京大) \\ \hspace{1zw} 長藤圭介(東京大)& \hspace{1zw}
%  \hspace{1zw}学\hspace{1zw}東京 学(西大)& [日本語著者名:明朝体10pt]\\
 % ※協賛・後援団体の会員資格で発表される場合は「正・学」は不要です。
 \end{tabular}
 % &\\
 \vspace{1zh} \\
 \begin{tabular}{l}
{\small Hisaaki IIDA, University of Tokyo, iida@dragon.t.u-tokyo.ac.jp}\\
 {\small Moju ZHAO, University of Tokyo, }\\
 {\small Junichiro SUGIHARA, University of Tokyo,}
 ~{\small Kazuki SUGIHARA, University of Tokyo}\\
 {\small Haruki KOZUKA, University of Tokyo,}
 ~{\small Jinjie LI, University of Tokyo}\\
 {\small Keisuke NAGATO, University of Tokyo,}\\
\end{tabular}
}
\makeatother

\abstract{ \small 
Recently, as the expectations for aerial robots to perform complex tasks increase, perching has been studied as a method to reduce power consumption during stationary phases by enabling the robot to attach to environmental objects, thereby extending the operational time of power-intensive aerial robots.
However, many existing approaches mount the attachment mechanism on the top or bottom of the robot, which severely restricts the attachment direction and flight posture to prevent the robot from tipping over or coming into contact with the environment and objects.
Additionally, many of these robots are designed specifically for perching and cannot perform other tasks.  
To address these issues simultaneously, we propose the development of an end-effector capable of both lateral perching and grasping a variety of objects and tools.
However, no previous research has successfully achieved both stable lateral perching and precise object grasping using a rotorcraft. Furthermore, implementing perching and transitioning back to flight in this study requires specialized flight control techniques.  
Therefore, this study aims to design a three-fingered tendon-driven hand powered by a single actuator as an end-effector capable of stable multi-directional perching and precise object grasping. Additionally, we will develop the motion planning for an aerial robot equipped with this hand.
}

\date{} % 日付を出力しない
\keywords{aerial robot, mechanical hand, tendon driven, perching, grasping}

\maketitle
\thispagestyle{empty}
\pagestyle{empty}

\small
\section{序論}
これまで、制約の少ない空路を利用した物体搬送や工場設備等の監視業務用インフラとして飛行ロボットを利用する企業や研究が現れ、飛行ロボットの社会進出が進んできた。近年、飛行ロボットに期待される役割はより複雑化し、飛行ロボットが単体で実行する空撮などのタスクから、外部の物体や環境との物理接触を含む高度な作業へ移行している。特に高所での作業や電線工事など、人間の作業員が行っていた危険作業を飛行ロボットに代替させる事例が増えている。複雑タスクの実行可能化に関して、複数の劣駆動機体による協調作業を扱った研究が行われている。しかし複数機体間での連携は単一機体の場合に比べて困難かつ不安定であり、機体の導入にも多くのコストがかかるという問題がある。自由度が高い全駆動機体を用いた研究も行われているが、飛行や作業に伴って消費する電力量が大きくなるため、長時間の駆動が難しくなるという問題がある。そこで本研究では、全駆動飛行ロボットの安定した機体制御性と単一機体での複数タスク実行能力を両立しつつ、消費電力を低減させることで長時間駆動を可能にすることを目指す。

消費電力の低減に関して、環境物体に結合し自重を支持することでロータの推力を用いず空中に留まるパーチングが研究されている。現状では機体の上部または下部に手先機構を持つものが多くを占めるが、結合目標位置周辺に突出部や壁面が存在する場合では機体が環境物体と接触するリスクがあるため、両者ともに結合時のアプローチ方向や結合位置の制限を強く受ける。よって、本研究では手先機構を機体に対して水平に配置し、横方向から結合アプローチを行う手法を用い、結合から離陸、解離までの機体制御手法をペンデュラム・パーチングとして提案する。

単一機体による複数タスク実行に関して、既存研究では予め特定タスク専用のツールを組み込んだ機体を使用するものが多く、タスク数に応じてペイロードが圧迫される。この問題はアタッチメント交換方式により解決可能であり、事前に予測しえない形状のツールや物体の把持が可能な飛行ロボットの研究がGripper Droneとして進められている。先述したパーチングのための手先機構とGripperをともに飛行ロボットに搭載することで、消費電力の低減と複数タスクの実行を両立可能であると考えられる。しかしこれらのハンドの機能が独立している場合では、パーチング機構が飛行中に無駄なペイロードとなるなどの問題があり、最適な設計とは言えない。よって本研究では、パーチングと物体把持をともに行えるハンド機構を開発し、飛行ロボットに搭載する。



\section{ハンドの設計}

\section{飛行動作設計}

\section{実験}

\section{結論と今後の展望}

\footnotesize
% \begin{thebibliography}{99}
\bibliographystyle{junsrt}
\bibliography{bib.bib}
% \bibitem{Shinjuku98}
% 新宿大五朗,渋谷次郎,東京 学,``キャスティングマニピュレーションに関する研究(第1報,可変長の紐状柔軟リンクを有するマニピュレータの提案とそのスイング制御法)'',{\it 機論C編}, vol.64-626, pp.3854--3861, 1998.

% \bibitem{Shinjuku99}
% Shinjuku, D., Shibuya, J. and Tokyo, M., ``Swing Motion Control of Casting Manipulation,'' {\it IEEE Control Systems}, vol.19-4, pp.56--64, 1999.

% \end{thebibliography}

\normalsize
\end{document}
